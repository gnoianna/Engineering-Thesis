\chapter{Podsumowanie}
\label{cha:podsumowanie}
W ostatnim rozdziale przedstawiono ogólne podsumowanie pracy. Zostaną w nim także zweryfikowane początkowe założenia. Rozdział zawiera opis możliwego rozwoju programu stworzonego w ramach owej pracy inżynierskiej.

\section{Weryfikacja początkowych założeń}
W pierwszym rozdziale owej pracy (sekcja \ref{sec:zalozeniaProjektu}) wymieniono założenia, które w etapie końcowym powinien spełniać stworzony projekt. Poniżej zostanie przedstawiona ich analiza oraz weryfikacja osiągniętych celów:

\begin{itemize}
    \item Działanie algorytmu miało zostać powiązane z analizą punktów charakterystycznych twarzy, co w stu procentach osiągnięto. Algorytm rozpoczyna działanie od ich wykrycia. Na tej podstawie odbywają się kolejne etapy, które dotyczą modyfikacji poszczególnych elementów.
    \item Samą identyfikację punktów udało się zrealizować z wykorzystaniem biblioteki dlib, dzięki której odbywa się wykrycie twarzy oraz jej istotnych elementów. 
    \item Po stworzeniu projektu algorytm przetestowano, według planu, na gotowym zbiorze danych, dzięki czemu można było sprawdzić poprawność jego działania.
    \item W ramach części praktycznej zaimplementowano prostą aplikację webową, wykorzystaną do walidacji algorytmu - według oczekiwań. Moim osobistym, początkowym założeniem było utworzenie filtra animującego awatar w czasie rzeczywistym. Nie udało się tego zrealizować z powodu długiego czasu działania funkcji \textit{warp()} z biblioteki scikit-image. Jednak jest to kwestia, którą w przyszłości warto rozwinąć. 
\end{itemize}

Wszystkie założenia postawione na początku tworzenia programu zostały zrealizowane. Stworzenie poprawnie działającego algorytmu wymaga dużo czasu oraz zaangażowania. Z tego powodu aktualna wersja programu nie jest idealna i z pewnością można ją dopracować. Możliwe ścieżki rozwoju zostaną opisane w następnym rozdziale.

\section{Możliwy rozwój projektu}
Działanie algorytmu zaimplementowanego w trakcie tworzenia tej pracy dyplomowej można usprawnić na wiele sposobów. Uważam, że niektóre etapy wymagają udoskonalenia, co zdecydowanie poprawiłoby osiągane rezultaty. Pierwszą ważną cechą, jest próba rozszerzenia programu o radzenie sobie z sytuacjami, gdy usta awatara się rozszerzają - warto zastanowić się nad kwestią uzębienia. 

Ponadto dotychczasowy czas działania niektórych funkcji wykorzystanych w programie zdaje się być zbyt długi. Lepszym rozwiązaniem byłoby zastąpienie ich wydajniejszymi modułami. Możliwe, że umożliwiłoby to jednoczesne udoskonalenie aplikacji - zakładka real-time mogłaby działać tak, jak na początku zakładano.

Uważam również, że etap nałożenia maski warto przemyśleć i być może spróbować rozwiązać w inny sposób. W tym momencie maska dopasowywana jest poprzez wykorzystanie funkcji z biblioteki scikit-image, która transformuje obraz na podstawie punktów charakterystycznych. Rezultat wydawał się zadowalający, jednakże podczas testów dwie osoby zwróciły uwagę na brak animacji brwi, co w tym momencie nie jest możliwe, ze względu na to że owe punkty biorą udział w transformacji. 

Niewykluczone, że dobranie modelu z inną ilością punktów charakterystycznych również wpłynęłoby na poprawę efektów jakie zapewnia działanie algorytmu.

Uważam, że wszystkie wymienione wyżej propozycje poprawiłyby efekty rezultatów, które w tym momencie można osiągnąć używając algorytmu animacji awatara. Warto również rozwinąć aplikację stworzoną na potrzeby ewaluacji, dokładniej przetestować jej działanie i popracować nad interfejsem graficznym.
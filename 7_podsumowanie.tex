\chapter{Podsumowanie}
\label{cha:podsumowanie}
W ostatnim rozdziale pracy zostanie przedstawione podsumowanie stworzonego projektu dyplomowego. Zostaną także zweryfikowane początkowe założenia. Zawarto również opis możliwego rozwoju programu stworzonego w ramach owej pracy inżynierskiej.
\section{Wnioski}
Nooo ze sie udalo stworzyc algorytm w sumie, ze roznie dziala dla danego awatara, ze tam wyszlo wlasnie w ankietach
no ale ze to wlosy ze to tamto siamto

\section{Weryfikacja początkowych założeń}
Założenia wymienione w sekcji \ref{sec:zalozeniaProjektu}
No to jest za pomoca punktow charakterystycznych
Jest użyta biblioteka dlib
Przetestowano na obrazach
Stworzono prosta aplikacje webowa -> co do niej no to ze miala byc barziej skomplikowana tzn real-time ale z powodu tego dzialania biblitoeki ze sie nie dalo w sumie

\section{Możliwy rozwój projektu}
Próba dobrania lepiej działających funkcji tych co robią warp
Dodanie tych zębów
Dobór modelu lepszego, albo inne dopasowywanie tej maski, to końcowe warp, ze przez to srednio te brwi sie ruszaja np i ze to juz zostalo wspomniane w tych ankietach
% W sumie tutaj można napisać o tym, że planem było stworzenie aplikacji animującej awatara real-time. Ale probelem okazał się czas tej funkcji warp.. Ze względu na to zdecydowałam się na inną formę tej apki itd itp.

% Rozwój no to te zęby np i oczy czy coś, no i to real-time działanie
% Udoskonalenie algorytmu czy coś, żeby był efektywniejszy

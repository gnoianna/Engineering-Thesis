\chapter{Przegląd istniejących zastosowań}

Niemal każdy z nas na co dzień używa smartfona i korzysta z różnego rodzaju aplikacji, których celem jest komunikacja z ludźmi, zarządzanie finansami, tworzenie notatek, dokumentów bądź też zapewnienie rozrywki użytkownikowi. W czasach, w których media społecznośiowe odgrywają znaczącą rolę popularne stało się stosowanie filtrów modyfikujących twarz, sylwetkę lub dodających do zdjęć efekty specjalne. 

Zgłebiając zasoby internetowe łatwo natrafić na narzędzia, które stosują rozwiązania zbliżone do implementowanego w niniejszej pracy algorytmu.

W tym rozdziale zostanie opisana technika deepfake, która z dnia na dzień zyskuje na popularności. Zostaną również przedstawione przykładowe aplikacje mobilne implementujące ową technikę i oparte o wykorzystanie sztucznej inteligencji w przetwarzaniu obrazów. 

% %---------------------------------------------------------------------------

\section{Technika deepfake}
Technologia deepfake umożliwia przedstawienie dowolnej osoby jako uczestnika danego filmu, czy też postaci znajdującej się na konkretnym zdjęciu. Jej celem jest zamiana wypowiedzi jednej osoby, na wypowiedź innej, to samo dotyczy ruchów ciała. Istnieje także możliwość spreparowania dźwięku, przez co mamy wrażenie, że słyszymy słowa wypowiadane przez znajomą nam osobę, tymczasem jest to fikcja uzyskana z pomocą sztucznej inteligencji. Owa technika to symulacja rzeczywistości, często wykorzystywana we współczesnym kinie w przypadku generowania komputerowych scenografii.

Kiedy podczas kręcenia siódmej części filmu Szybcy i Wściekli zmarł Paul Walker kierownicy produkcji musieli zmierzyć się z niemałym na tamte czasy wyzwaniem, symulując sceny z jego udziałem. W dzisiejszych czasach technologia deepfake niezwykle się rozwinęła i stała się bardzo popularna, każda osoba ma możliwość wygenerowania filmu będącego deepfake'iem w kilka minut.

Słowo deepfake to tak naprawdę połączenie dwóch angielskich wyrazów. Nawiązuje ono do technologii uczenia głębokiego (ang. deep learning), z którą owa technika jest ściśle związana, oraz do słowa fake oznaczającego coś nieprawdziwego, udającego inną rzecz.

Działanie algorytmu deepfake często oparte jest na wykorzystaniu autoenkoderów lub sieci GAN (ang. Generative Adversarial Network). Według wielu osób wspomniane wyżej sieci mogą być związane z ogromnym rozwojem tejże technologii. Podobizny, które zostają wygenerowane poprzez ich zastosowanie wydają się być prawie nieodróżnialne od rzeczywistych twarzy \cite{deepfake}.

Stworzenie filmu określanego jako deepfake musi zostać poprzedzone przez odpowiednie procedury. Na przykładzie enkoderów pierwszym krokiem jest przepuszczenie zdjęcia źródłowego jak i docelowego przez enkoder, który uczy się podobieństw między dwoma twarzami. Następnie dane są kompresowane, wyciągane są wspólne cechy obydwu zdjęć. Kolejnym etapem jest wytrenowanie dwóch dekoderów, które zostaną użyte to odzyskania twarzy każdej z osób. Ostatnim krokiem jest podmiana twarzy między dekoderami. Dekoder ma za zadanie zrekonstruować twarz drugiej osoby na podstawie orientacji pierwszego obrazu. 

Sieci GAN działają inaczej, na początku zostają trenowane poprzez kilkugodzinne analizowanie rzeczywistego filmu wideo. Dzięki temu są w stanie nauczyć się jak wygląda twarz osoby pod różnymi kątami, w różnym świetle. Następnie łączy się wytrenowaną sieć z technikami grafiki komputerowej, nakłada się kopię osoby na danego aktora. Wszystko odbywa się poprzez mapowanie odpowiednich punktów charakterystycznych twarzy.

Technologia deepfake niesie ze sobą niemałe zagrożenie. Poziom zaawansowania sprawia, że czasami ciężko na pierwszy rzut oka wykryć film, na którym została zastosowana. Na dzień dzisiejszy dzięki dokładniejszej analizie takowego wideo jesteśmy w stanie wykryć czy stworzony film jest fikcją. Jednak według wielu przypuszczeń, niewykluczone, że wraz z rozwojem sztucznej inteligencji przestanie to być możliwe.

% %---------------------------------------------------------------------------

\section{Dostępne aplikacje mobilne}

\subsection{Wombo.ai}
Wombo.ai \cite{womboai} to bardzo popularna, darmowa aplikacja mobilna, która została stworzona w celu rozrywkowym. Jest ona dostępna zarówno na smartfony jak i tablety z systemem operacyjnym Android i IOS, co zdecydowanie jest jej walorem. Reprezentowaną przez nią ideą jest tworzenie zabawnych filmików, na których ożywiane są wgrane wcześniej fotografie. Zaletą tej aplikacji jest jej prostota i łatwość w obsłudze. 

Wszystko odbywa się w kilku krokach, należy zrobić zdjęcie swojej twarzy albo wgrać takowe z galerii, następnie wybrać utwór muzyczny na podstawie którego otrzymujemy krótkie nagranie ze stworzoną animacją. Filmiki zostają wygenerowane z użyciem technologii deepfake przez co są bardzo realistyczne. Jej działanie powiązane jest z wykorzystaniem sztucznej inteligencji do synchronizowania ruchu.

% %---------------------------------------------------------------------------

\subsection{Anyface: face animation}
Anyface \cite{anyface} to aplikacja mająca na celu, podobnie jak poprzedniczka, zapewnienie rozrywki użytkownikowi. Jest ona dostępna w wersji mobilnej tylko na Androida.

W odróżnieniu do wspomnianego wyżej programu to narzędzie posiada dużo więcej funkcji, przez co może sprawiać wrażenie bardziej atrakcyjnego dla użytkownika. Poza animacją zdjęcia na podstawie wybranej frazy istnieje możliwość dodania własnego nagrania, z którego użyciem zostanie wygenerowana animacja. Narzędzie posiada także moduł edycji i udoskonalania zdjęć, odbiorca jest w stanie dodawać filtry, efekty specjalne i inne obiekty, które nadadzą animacji unikalny charakter. 

% %---------------------------------------------------------------------------

\subsection{MotionPortrait}
MotionPortrait \cite{motionportrait} to bardzo podstawowa aplikacja mobilna dostępna zarówno na Androida jak i system IOS. Kategoria, w której można ją znaleźć to rozrywka. Posiada opcję zrobienia własnego zdjęcia, dodania zdjęcia z galerii lub wyboru udostępnionej fotografii. Po wykonaniu tego etapu mamy możliwość animacji naszej podobizny poprzez wybór zdefiniowanych wyrazów twarzy. 

Dzięki połączeniu z mikrofonem w przypadku wypowiadania jakichkolwiek słów nasza twarz jest animowana. Animacja dotyczy tutaj tylko ust, awatar mruga regularnie oczami. Efekt, który otrzymujemy nie jest zbyt satysfakcjonujący. Na zdjęcie możemy też nałożyć różne filtry, dodać efekty. Ostatecznie istnieje możliwość nagrania krótkiego wideo, które potem możemy zapisać.

% %---------------------------------------------------------------------------

\section{Wnioski}
Na rynku dostępna jest duża ilość aplikacji podobnych do programów opisanych powyżej. Niektóre z nich są bardzo podstawowe a efekty osiągane przez te narzędzia nie wydają się być spektakularne. Istnieją jednak rozwiązania, które odstają od innych swoją dokładnością działania i efektowną realizacją. 

Pewne jest, że użytkownicy chętnie korzystają z owych aplikacji. Przeglądając media społecznościowe na każdym kroku możemy się natknąć na filmiki stworzone z użyciem techniki deepfake. Co więcej, narzędzia oparte na tych technologiach niezwykle szybko się rozwijają i zaczynają zaskakiwać swoimi opcjami.

Powyższe fakty sprawiają, że stworzenie algorytmu animacji awatara w ramach tejże pracy inżynierskiej wydaje się być sensowne. Są to narzędzia, które wykorzystuje się nie tylko w celu rozrywkowym, ale także w dziedzinie kinematografii. 



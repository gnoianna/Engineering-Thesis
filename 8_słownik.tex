\chapter{Opis używanych pojęć}
% \label{}
% \begin{itemize}
%     \item punkty charakterystyczne (ang. landmarks) - punkty rozpoznawcze twarzy wykrywane na obrazie. W kontekście tejże pracy, używane w celu lokalizacji i reprezentacji najistotniejszych obszarów ludzkiej twarzy. Mowa tutaj o linii szczęki, ust, nosa, oczu i brwi.
%     \item awatar - tożsamość internetowa, czyli cyfrowa postać odzwierciedlająca wizerunek danej osoby. Awatary używane są w grach komputerowych, na forach i w mediach społecznościowych. Zazwyczaj przybierają dowolną formę (człowiek, zwierzę). W ramach niniejszej pracy jako awatar rozumie się nierealną cyfrową postać mającą formę osobową. Ważną kwestią jest prostota takiego charakteru - wyraźne rysy twarzy, brak cieni.
%     \item predictor - w kontekscie wykrywania twarzy, model glebokiego uczenia
%     \item detector - to samo, co to jest w kontekscie tego algorytmu
%     \item triangulacja - metoda podzialu figury geometrycznej itd itp
%     \item sympleksy
%     \item maska - tutaj kwestia zapytania, trzeba ogarnac poprawna definicje i sprawdzic czy dobrze jej uzywam
% \end{itemize}
\chapter{Ewaluacja i rezultaty}
\label{cha:ewaluacjaIRezultaty}

\section{Walidacja działania aplikacji}
% No to tutaj może dla jakichś 15 osób czy coś, poprosić o testowanie i np. osobno te ze zdjęciami a osobno real-time
% szczęście,
% smutek,
% strach/zaskoczenie,
% gniew/obrzydzenie
% No i ocenić to może w skali, czy rezultaty są wystarczające
\section{Ocena osiągniętych efektów}
Noo i tutaj sprawdzić np czy działa dla tego szalonego modelu, jeśli tak to sprawdzić jak tam efekty działania aplikacji dla obydwu, można np zestawić z dwóch modeli i kazać ocenić ludziom. Plus jakaś tabelka przedstawiająca porównanie czasu przetwarzania dla kilkunastu obrazów.

No i we wnioskach to na pewno:
1. Jak te modele wypadają na swoim tle, tzn który lepszy i poprzeć wynikami z ankiety
2. Przedstawić ocenę aplikacji tej strony real time vs tej z obrazami dwoma
3. No i pewnie we wnioskach ze gorzej ta real time dziala, no i ze pewnie trzeba dopracowac i ze tez ma wplyw na to oswietlenie, jakies ustawianie zle twarzy bla bla
